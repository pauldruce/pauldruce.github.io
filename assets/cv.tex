\documentclass[10pt]{article}

\usepackage{hyperref}
\usepackage{geometry}
\usepackage[parfill]{parskip}
\usepackage{lmodern}
\usepackage[utf8]{inputenc}
\usepackage{enumitem}
\usepackage{titlesec}
\usepackage{fancyhdr}
\usepackage{multicol}
\usepackage{tabularx}
\usepackage{xcolor}

\def\name{Paul Druce}
\providecommand{\tightlist}{%
  \setlength{\itemsep}{0pt}\setlength{\parskip}{0pt}}

% Define colors
\definecolor{sectioncolor}{RGB}{0, 0, 128}  % Dark blue
\definecolor{linkcolor}{RGB}{0, 102, 204}   % Bright blue

% Hyperref settings
\hypersetup{
  colorlinks = true,
  urlcolor = linkcolor,
  linkcolor = linkcolor,
  pdfauthor = {\name},
  pdfkeywords = {},
  pdftitle = {\name: Curriculum Vitae},
  pdfsubject = {Curriculum Vitae},
  pdfpagemode = UseNone
}

\geometry{
  body={6.5in, 9.0in},
  left=1.0in,
  top=1.0in
}

% Disable numbers on sections.
\setcounter{secnumdepth}{0}

% Section formatting
\titleformat{\section}{\color{sectioncolor}\normalfont\Large\bfseries}{}{0em}{}
\titleformat{\subsection}{\color{sectioncolor}\normalfont\large\bfseries}{}{0em}{}


\begin{document}

{\huge \name}

\subsection{Current responsibilities}\label{current-responsibilities}

\textbf{Senior Software Engineer in Test, The MathWorks}
\hfill \textbf{May 2023 - Current} \\
Help my teams (Cloud Platform Integrations and Modelscape) to implement
comprehensive CI pipelines.

Skills learnt:

\begin{itemize}
\tightlist
\item
  Kubernetes based web app development.
\item
  Deployment of modern web app on Azure and AWS.
\item
  Managing software release pipelines.
\item
  Golang development and testing.
\item
  Refactoring existing codebases to align with best practices
\end{itemize}

\textbf{Software Engineer in Test, The MathWorks}
\hfill \textbf{May 2022 - May 2023} \\
As part of the Cloud Platform Integrations team at MathWorks, I worked
on integrating MATLAB in Jupyter notebooks.\\
Also worked on Modelscape - a new offering from The MathWorks aimed at
helping financial institutions manage their models.\\

Skills learnt:

\begin{itemize}
\tightlist
\item
  JavaScript and TypeScript development and testing.
\item
  Python development and testing.
\item
  Build and test pipelines in TeamCity and GitHub Actions
\item
  Docker for environment reproducibility and testing.
\item
  Automated GUI testing using Playwright framework.
\end{itemize}

\subsection{Skills}\label{skills}

\begin{itemize}
\tightlist
\item
  Experienced in high performance computing and commercial software in C
  and C++.
\item
  Fluent in the following programming languages: C, C++,
  JavaScript/TypeScript, Golang, Python and Bash.
\item
  Experienced using Git and Perforce for collaborative development.
\item
  Experienced setting up CI pipelines using GitHub Actions and TeamCity.
\item
  Experience using Linux, macOS and Windows development environments.
\item
  Internet of Things development using ESP8266/ESP32 and other devices,
  i.e., Arduino framework, ESP-DF framework, micropython as well as how
  to use PlatformIO.
\item
  Experience in data analysis using Python and Mathematica.
\item
  Experienced using LaTex (including TikZ and beamer) and the Microsoft
  Office suite.
\item
  Web development, including animations; HTML/CSS/JavaScript.
\end{itemize}

\subsection{Previous Professional
Experience}\label{previous-professional-experience}

\textbf{Consulting Engineering at MachineWorks}
\hfill \textbf{May 2021 - April 2022} \\
I aided commercial 3D software developers to implement and use
MachineWorks software for a wide variety of applications, using the
languages C, C++ and C\#.

I also create demonstrative applications to show case MachineWorks
software in new markets.\\
This includes the creation of new workflows and GUI applications.\\
I also develop algorithms and create example code for customers to use
in their products to facilitate their products to market faster.

\textbf{Academic Tutor and Mentor} \hfill \textbf{Aug 2017 - May 2023 }
\\
I am an experienced tutor in mathematics and physics at the levels of
university, A level and GCSE. \\
I also mentored students for various mathematical entrance exams and the
UKMT mathematical competitions.

I have vast experience in both face-to-face tutoring and online tutoring
via shared online whiteboards. I prepared lessons and question sheets at
appropriate levels for my tutees. My role was to build the confidence
and the abilities of my students and provide them with a comfortable
environment to ask any questions they may have.

\subsection{Education}\label{education}

\textbf{PhD in Mathematical Sciences, University of Nottingham, UK:}
\hfill \textbf{2015 -2020}\\
\emph{Title: Spectral Geometry of Fuzzy Spaces} \\
\emph{Supervisor: Prof.~John Barrett}

I developed the understanding of finite non-commutative geometries and
how they might be useful in the theory of quantum gravity. \\
\textbf{Mathematical areas studied:} Differential geometry,
non-commutative geometry, quantum geometry, representation theory. \\
\textbf{Key skills developed:} independent working, public speaking,
data analysis using Python, knowledge of Monte Carlo simulations

For my PhD I worked on an area of mathematics called non-commutative
geometry and how it might be useful in the theory of quantum gravity. My
research was to investigate the use of called \emph{fuzzy spaces} as
candidates for quantum spacetimes. I investigated the dimension and
volume of these spaces by analysing the spectrum of the Dirac operator.
I also investigated the role of Lie group symmetries in restricting the
fuzzy spaces possible.

In pursuit of my research I developed my knowledge of data analysis
using the language Python. I also became familiar with the workings of
Monte Carlo simulations and the application of machine learning to
physical problems. A topic I am very interested in pursuing further.

\textbf{First Class Masters in Mathematics and Physics, University of
Warwick, UK}: \hfill \textbf{2011-2015}\\
\emph{Masters Dissertation: Multiferroicity Emerging from Frustrated
Spin Interactions} \\
\emph{Supervisors: Prof J. Staunton and Dr J. Lloyd-Hughes}\\

Year 1: 69.3\%, Year 2: 82.8\%, Year 3: 82.9\%, Year 4: 83.0\%, Overall
grade: 81.6\%

During my time at Warwick, I studied a wide range of mathematics and
physics topics. My interests were in both the abstract mathematics and
the physics of matter and its constituents. Here is a list of topics I
studied during my time at Warwick: Real Analysis, Differential
Equations, Groups and Rings, Complex Analysis, Classical Mechanics,
Statistical Physics, General Relativity, Solid State Physics, Fluid
Dynamics and many more. I maintained an average grade of \(81\%\)
throughout my course achieving one of the best marks in the year. I
became proficient in the programming language C and its use in
high-performance computing. I learnt how to implement parallel computing
by making us of OpenMP and MPI frameworks.

\textbf{A Levels, King Edward Vi Sixth Form, Sheffield, UK:}
\hfill \textbf{ 2009 - 2011 }\\
Maths: A*, Physics: A, Chemistry: A.

\textbf{GCSEs, Birley Community College, Sheffield, UK:}
\hfill \textbf{ 2004 - 2009 }\\
9 A's, 2 B's and one C.

\subsection{Teaching Experience}\label{teaching-experience}

\begin{itemize}
\item
  \textbf{UKMT Volunteer
  Mentor}\hfill \textbf{ October 2020 - May 2022 }

  I was a volunteer for the UK Mathematics Trust. My role involved
  mentoring the students who are entering the advanced UKMT competitive
  mathematics exams. I was also involved in various outreach projects
  that the UKMT organise. I aided with the creation of appropriate
  questions and aided in the distribution of content via social media.
\item
  \textbf{ PhD Demonstrating and Marking}
  \hfill \textbf{ Oct 2015 - Jul 2019}

  Alongside my PhD, I was employed to help students with their questions
  for various undergraduate mathematics modules for Engineering
  students, Physics students and Mathematics students. This includes
  courses such as Introduction to Mathematical Physics (2nd year),
  Differential Equations and Fourier Analysis (2nd year), Applied
  Mathematics (1st year) and more. I also mark the mid-term exams and
  coursework for various modules.

  \begin{itemize}
  \tightlist
  \item
    Introduction to Mathematical Physics (2nd year)
    \hfill \textbf{ 2015-2019}
  \item
    Mathematics for Physics and Astronomy (1st year)
    \hfill \textbf{ Full year 2015-2019}
  \item
    Calculus and Linear Algebra (1st year for Natural Science students)
    \hfill \textbf{Full year 2018-2019}
  \item
    Applied Mathematics (1st year)\hfill \textbf{Full year 2017 - 2018}
  \item
    Mathematical Analysis (2nd year) \hfill \textbf{ Autumn 2018}
  \item
    Differential Equations and Fourier Analysis (2nd year)
    \hfill \textbf{Spring 2017}
  \item
    Fluid Dynamics (3rd year) \hfill \textbf{Spring 2016 }
  \end{itemize}
\item
  \textbf{ Non-commutative Geometry Seminar Series for Master Students}
  \hfill \textbf{ Oct 2016 - Apr 2017}

  I organised and delivered a seminar on topics surrounding my research
  over the course of two semesters. The seminar was delivered to masters
  students at the University of Nottingham. The was regularly attended
  by around 10 students despite it not being for credit towards their
  degree. The course resulted in two students choosing noncommutative
  geometry as the subject for their masters dissertation, and one of the
  students chose to undertake a PhD with my supervisor. Some lecture
  notes suitable for a UK masters student were developed on the
  foundational results of non-commutative geometry.
\item
  \textbf{Undergraduate Revision Classes} \hfill \textbf{ 2012-2015}

  As part of the Warwick Physics Society:

  \begin{itemize}
  \tightlist
  \item
    I updated and maintained revision guides
  \item
    I organised and ran revision lectures on various topics
  \item
    I provided workshops to run alongside a university course. These
    were aimed at first time programmers in C programming, to help them
    understand the language and develop the programming skills
    necessary.
  \end{itemize}
\end{itemize}

\subsection{Academic Activities}\label{academic-activities}

\subsubsection{Research Interests}\label{research-interests}

My research interests span a wide area within Mathematical Physics. I am
deeply interested in the mathematical description of the universe with
emphasis on the precise nature of spacetime. With my recent work
investigating the use of noncommutative geometry to model spacetimes
with a high energy cutoff. I also interested in any novel use of
mathematics in physical situations, such as the use of topology and
algebraic methods in condensed matter studies.

\subsubsection{Outreach}\label{outreach}

I am an enthusiastic advocate for mathematics and science. I was on the
organisational committee for the international festival Pint of Science
2019. I helped organise the Nottingham branch where academics from the
University of Nottingham go to the pubs of Nottingham to explain their
current research to the public in an understandable manner. I am always
looking for ways to bring mathematics and science out of the
universities and into public view.

\subsubsection{Past Research Projects}\label{past-research-projects}

\textbf{PhD Research Project - Spectral Geometry of Fuzzy Spaces}
\hfill \textbf{2015-2019}

\emph{Keywords:} Noncommutative geometry, Monte Carlo simulations,
spectral geometry, quantum gravity, symmetries in physics

My PhD research was concerned with the use of finite noncommutative
geometries as candidates for quantum spacetimes. These so-called
\emph{fuzzy spaces} possess an energy cutoff whilst retaining Lie group
symmetries. My PhD research was to investigate the dimension and volume
of these spaces by analysing the spectrum of the Dirac operator. I also
investigated the role of Lie group symmetries in restricting the
possible fuzzy spaces possible.

\textbf{Masters Research Project - Multiferroicity Emerging from
Frustrated Spin Interactions} \hfill \textbf{2014-2015}

\emph{Keywords:} Solid-state physics, Theoretical physics, mathematical
modelling, frustrated systems, quantum mechanics.

During the Masters' year of my undergraduate degree, I undertook a
research project, supervised by Prof.~J. Staunton and Dr.~J.
Lloyd-Hughes, in which we investigated the various spin configurations
of materials with a multiferroic phase. We aimed to identify what was
special about these materials spin configurations that caused them to
possess a multiferroic phase. This project was conducted with the aim of
aiding the design of high temperature multiferroic devices. We
specifically studied Cupric Oxide (CuO) which has a multiferroic phase
between 213K and 230K, primarily by using mean field model. This project
resulted in a functional model which predicted a multiferroic phase at a
temperature which is in reasonable agreement with the experimental
values.

\textbf{Undergraduate Summer Research - Knotted Nematics}
\hfill \textbf{August 2014-September 2014}

\emph{Keywords:} Condensed Matter, Mathematical Physics, Liquid
Crystals, statistical physics

This was funded under the Undergraduate Research Scholarship Scheme and
was supervised under G. P. Alexander at The University of Warwick. The
main aim of this research was to develop a construction for describing
knotted liquid crystals, specifically knotted nematics. We improved upon
existing ideas that used Milnor's Fibration Theorem and developed a
method to construct the necessary complex polynomial for Milnor's
theorem, for a given knot.

\subsubsection{Publications}\label{publications}

\emph{Spectral estimators for finite non-commutative geometries}.
Barrett, J., Druce, P., Glaser, L.: J Phys Math Theor. 52, 275203
(2019). doi:10.1088/1751-8121/ab22f8

\subsubsection{Grants and Awards}\label{grants-and-awards}

\begin{itemize}
\tightlist
\item
  \emph{Non-commutative Geometry and Quantum Gravity} EPSRC Studentship,
  funding my PhD studies, hosted by the University of Nottingham
  (September 2015 - September 2018)
\item
  \emph{Knotted Nematics} - funded as an Undergraduate Research
  Scholarship Scheme by the University of Warwick, and supervised by Dr
  G. P. Alexander (August-September 2014).
\end{itemize}

\subsubsection{Talks}\label{talks}

\begin{itemize}
\tightlist
\item
  \emph{Noncommutative Geometry and Gravity Models} Talk given at
  Collabor8.2 meeting at Lancaster University, UK, May 2018. Slides
  here. \\
\item
  \emph{Fuzzy Geometries and Spectral Zeta Functions.} Invited by Lisa
  Glaser at Radboud University, Netherlands, April 2017. Slides here. \\
\item
  \emph{Algebraic Knots and Liquid Crystals}. At the Warwick Imperial
  Autumn Meeting, 2014 (University of Warwick, UK, November 2014).
  Slides here. \\
\item
  Poster presented at Quantum Gravity on the Computer conference March
  2018. PDF here.
\end{itemize}

\subsubsection{Conferences Attended}\label{conferences-attended}

\begin{itemize}
\tightlist
\item
  Gauge Theories and Noncommutative Geometry - Nijmegen, April 2016,
  \url{http://www.noncommutativegeometry.nl/ncg2016/}
\item
  Quantum Structure of Spacetime - Belgrade, August 2016,
  \url{http://qssg16.ipb.ac.rs/}
\item
  Quantum Spacetime - Porto, January 2017,
  \url{https://www.fc.up.pt/quantumspacetime17/}
\item
  Talking Maths in Public - Bath, September 2017,
  \url{http://talkingmathsinpublic.uk/}
\item
  Quantum Spacetime and Physics Models - Corfu, September 2017,
  \url{http://www.physics.ntua.gr/corfu2017/qg.html}
\item
  Quantum Structure of Spacetime - Sofia, February 2018,
  \url{http://theo2.inrne.bas.bg/~dobrev/QST-18.htm}
\item
  Quantum Gravity on the Computer - Stockholm, March 2018,
  \url{https://agenda.albanova.se/conferenceDisplay.py?confId=6242}
\item
  Collabor8.2 - Lancaster, May 2018,
  \url{http://www.collabor8research.com/}
\item
  Physical Applications of Fuzzy Spaces - Brussels, January 2019
\end{itemize}

\subsection{Other professional
experiences}\label{other-professional-experiences}

\begin{itemize}
\item
  \textbf{Open Day Assistant at the University of Warwick}
  \hfill \textbf{September 2012 - July 2015}

  Throughout my undergraduate degree I was part of the open day team for
  the Physics department at Warwick university. My role included taking
  the prospective students on tours of the campus, informing them of
  important and interesting aspects of the university. I was also given
  the responsibility to present experiments to prospective students,
  engaging them with thought provoking questions. I was also part of the
  team to inform the prospective students about the courses available by
  the Physics department and answer any questions they may have about
  life at university.
\item
  \textbf{Administrative Worker at Split The Bills Ltd.}
  \hfill \textbf{Aug-September 2012}

  My roles in this temporary role was to communicate with the student
  registering for the service and then contact utility providers to
  setup the new accounts. Handling any issues that would arise in a
  timely and professional manner.
\item
  \textbf{Warehouse Operative at River Island} \hfill \textbf{2010-2011}

  I was part of the team that unpackaged new deliveries of clothes and
  prepare clothing to be presented on the store floor. This required
  adaptability as each delivery changed in size. As well as good team
  work and communication as the unpacking procedure was split in to
  various stages, with a separate person per stage. As well as working
  in an efficient manner. I was also responsible for searching and
  retrieving clothing requested by the store front.
\item
  \textbf{Voluntary Sale Assistant at British Heart Foundation}
  \hfill \textbf{ 2010-2011}

  My role at the British Heart Foundation included assisting customers
  in finding items, informing them about the charity and maintaining the
  store. As the British Heart Foundation has a wide range of customers
  and staff, I had to quickly learn to adapt my communication and sales
  approach to fit their needs and situation.
\end{itemize}

\end{document}
